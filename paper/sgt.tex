\section{Serialization Graph Testing}
\label{sec:sgt}

This section presents serialization graph testing. 
\Cref{sec:sgt-description} describes the conflict graph theorem and how it is used by SGT.
In~\Cref{sec:sgt-algo-adjust}, the algorithmic adjustments
made in~\cite{DBLP:conf/icde/Durner019} are given, before~\Cref{sec:sgt-many-core-opt} describes the 
many-core optimized graph data structure used in~\cite{DBLP:conf/icde/Durner019}, which serves as the basis for mixed serialization graph testing.

\subsection{Protocol Description}
\label{sec:sgt-description}

% Conflict graph theorem 
An execution of transactions can be represented by a \emph{schedule}, a time ordered 
sequence of their operations. For example, consider transactions $T_1$, $T_2$, and $T_3$ shown in
schedule $s$ below.
$$ s = w_1 [x] \ r_2 [x] \ r_2 [y] \ w_1 [y] \ w_2[z] \ w_3[z] \ r_3 [x] \ r_3 [a] \ w_4 [a] \ c_1 \ c_3 \ c_2 \ c_4 $$
This schedule can be represented by a \emph{conflict graph} $CG(s)$, shown in~\Cref{fig:conflict-graph}. 
Nodes represent transactions and conflicting operations $a_i$ of $T_i$ and $b_j$ of $T_j$ such 
that $a_i[x] < b_j[x]$, where $ \ T_i \neq T_j$, are represented by an edge $T_i \rightarrow T_j$;
possible conflict pairs are $(a,b) \in [(r,w),(w,r),(w,w)]$. For example, in $s$, $T_2$ reads $x$ 
after $T_1$ writes to $x$, thus there exists an edge from $T_1$ to $T_2$ in~\Cref{fig:conflict-graph}.
Changing the order of conflicting operations \emph{could} alter the behavior of at least one 
transaction. Therefore, an execution of transactions is conflict serializable if a serial ordering of
transactions that satisfies all conflict edges can be found. Such a serial ordering exists iff the 
conflict graph is acyclic. This is known as the \emph{conflict graph theorem}~\cite{DBLP:books/aw/BernsteinHG87}. 
Note, $s$ is not conflict serializable because $CG(s)$ in~\Cref{fig:conflict-graph} contains a cycle.

\begin{theorem}[Conflict Graph Theorem]
  A schedule $s$ is conflict serializable iff its corresponding conflict graph
  $CG(s)$ is acyclic.
\end{theorem}

\begin{figure}[htbp]
  \centering
    \begin{tikzpicture}[node distance=0cm]
      \node (t1) [vertex, xshift=0cm, yshift=0cm] {$T_1$};
      \node (t2) [vertex, xshift=2cm, yshift=0cm] {$T_2$};
      \node (t3) [vertex, xshift=4cm, yshift=0cm] {$T_3$};
      \node (t4) [vertex, xshift=6cm, yshift=0cm] {$T_4$};
      \draw [edge] (t1) -- (t2);
      \draw [edge] (t2) -- (t3);
      \draw [edge] (t3) -- (t4);
      \draw [edge] (t1) to  [out=-45,in=-135]  (t3);
      \draw [edge] (t2) to  [out=135,in=45]  (t1);
    \end{tikzpicture}
  \caption{Conflict graph representation of $s$.}
  \label{fig:conflict-graph}
\end{figure}

% SGT description 
SGT directly utilizes the conflict graph theorem by maintaining an acyclic conflict graph. 
For each operation, conflicts are determined and edges inserted into the graph. An important point 
to note is the orientation of edge insertions: a transaction only inserts edges incoming to \emph{itself}. 
After edge insertion, a cycle check is performed \emph{before} executing the operation. If 
executing the operation would introduce a cycle the offending transaction is aborted and its edges removed. At commit time, 
a final cycle check is executed, if no cycle is found the transaction commits and removes its edges. In short, 
SGT provides serializability by ensuring the acyclic invariant. 

% No false negatives 
% The SGT implementation in~\cite{DBLP:conf/icde/Durner019} is now described as this forms the basis for MSGT. 

\subsection{Algorithmic Adjustments} 
\label{sec:sgt-algo-adjust}

Due to space constraints nodes must be pruned 
from the conflict graph. The SGT algorithm 
sketched in~\Cref{sec:sgt-description} allows
transactions to commit with incoming edges if 
they pass a cycle check. Under this scheme 
simply deleting nodes 
of committed transactions can lead to subtle 
serialization violations.
Assume in~\Cref{fig:conflict-graph}, $T_3$ has 
committed and $T_2$ is active. If $T_3$ is 
removed
upon commitment, $T_2$ may subsequently perform 
an operation that introduces a cycle with $T_3$, 
but which goes undetected due to $T_3$'s removal, introducing a serialization error. To solve this, in~\cite{DBLP:conf/icde/Durner019} a transaction can commit iff it has no incoming edges, 
else delaying until this condition is met.
As transactions only insert edges incoming to themselves, after issuing a commit request, no more 
incoming edges are added. Thus, once all incoming edges are removed from the committing node, via the parent node aborting or committing,
it will not be in a cycle.

This rule also achieves two desirable properties: \emph{recoverability} and \emph{order preservation}. 
Recoverability ensures failures do not leave the database in an inconsistent state. 
In $s$, $T_1$ writes $x$, then $T_2$ reads $x$. If $T_2$ commits before $T_1$, $T_1$ may subsequently abort, at 
which point $T_2$ has read from a value that never existed. Delaying $T_2$'s commit until it has no incoming edges 
prevents this issue; if $T_1$ aborts then $T_2$ is also aborted. \emph{Order preservation} ensures the real-time commit order matches the 
serialization order. In $s$, $T_4$ overwrites the value $a$ read by $T_3$, thus
in the serial order: $T_3 \rightarrow T_4$. If $T_4$ commits before $T_3$ no recoverability issues 
are introduced, but the real-time commit order does not match the serialization order. Delaying $T_4$'s commit 
until it has no incoming edges avoids this, providing users with an improved, more intuitive, experience.

\subsection{Many-Core Optimizations}
\label{sec:sgt-many-core-opt}

% Graph data structure 
Many-core DBMSs use optimistic approaches as their pessimistic counterparts suffer from poor performance as the core count 
increases. A common bottleneck in optimistic approaches is an exclusive single-threaded verification phase. Regards SGT, 
in~\cite{DBLP:conf/icde/Durner019} to avoid a global lock for graph operations, a data structure with a node-local locking 
protocol is used. Nodes in the graph each store a transaction status (committed, active, aborted) and two sets of pointers representing 
incoming and outgoing edges. Nodes can then be locked in two 
modes:
\begin{itemize}
    \item \textbf{\emph{Shared mode:}} transactions can concurrently access the node for edge insertions, edge 
    deletions, and cycle checking. Edge sets guarantee thread-safe concurrent access for scans, insertions, and 
    deletes under the shared lock. 
    \item \textbf{\emph{Exclusive mode:}} used for the  commit-critical check for incoming edges. 
\end{itemize}

The protocol works as follows: when a transaction $T_x$ identifies a conflict with transaction $T_y$ it first
checks if an edge already exists from $T_y$ to $T_x$, if so, no additional work is needed. 
Else, $T_x$ acquires a shared lock on $T_y$'s node. If $T_y$ is active, then an edge pointer to $T_x$ is 
inserted into $T_y$'s outgoing edge set and an edge pointer from $T_y$ is inserted into $T_x$'s incoming edge 
set.
$T_x$ then checks for a cycle using a \emph{reduced depth-first search} (DFS)
algorithm. Reduced DFS begins at the validating node ($T_x$) and traverses only the portion of the graph that is needed; 
each step holds nodes in shared lock mode.
At commit time, $T_x$ acquires an exclusive lock on its node and checks for incoming edges. If there are none, 
$T_x$ commits and shared locks are acquired on each node in $T_x$'s outgoing edge set and the 
edge from $T_x$ removed. Else, there exists at least one incoming edge and the exclusive lock is released and the check is 
repeated. 
Another common bottleneck is the reliance on a global timestamp 
allocator, this is avoided in~\cite{DBLP:conf/icde/Durner019} by letting conflict graph nodes double up as transaction ids.

SGT requires a mechanism to derive conflicts.
In~\cite{DBLP:conf/icde/Durner019} rows store a sequential history of accesses. 
Each access stores the operation type, read or write, and the transaction id.
Decoupling the access information from the graph data structure decreases contention.
Note, sequentially ordered access is ensured by per-row spin-locks which are released immediately 
after the operation completes, i.e., the lock is not held until commit time, an improvement over lock-based approaches.

% \begin{table}[htbp]
    \centering
    \caption{Example table with access history \{transaction id, operation type\}.}
    \begin{tabular}{|c|c|c|l|}
    \hline
   Id   & Name    & Age   & \multicolumn{1}{|c|}{Access History}                  \\ \hline
   $x$  & Jack    & 27    & $\{1,w\} \rightarrow \{2,r\} \rightarrow \{3,r\} $    \\ \hline
   $y$  & Stuart  & 62    & $\{2,r\} \rightarrow \{1,w\} $                        \\ \hline
   $z$  & Holly   & 21    & $\{2,w\} \rightarrow \{3,w\} $                        \\ \hline
   $a$  & Heather & 57    & $\{4,w\} \rightarrow \{3,r\} $                        \\ \hline
    \end{tabular}
    \label{fig:conflict-detection}
\end{table}


In summary, the SGT implementation in~\cite{DBLP:conf/icde/Durner019} uses a commit rule to simplify node removal, 
decouples conflict detection, and employs a highly parallel graph structure that allows concurrent cycle checking. In 
particular, this protocol scales well as the commit critical check only shortly blocks other threads from accessing a node 
and only the part of the conflict graph needed for validation is traversed. Lastly, it minimizes unnecessary aborts, 
accepting all conflict serializable schedules and provides an ideal baseline for the development of MSGT. 
