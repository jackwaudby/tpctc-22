\section{Mixing Theory}
\label{sec:mixing-theory}

This section presents the correctness criteria utilized by MSGT. \Cref{sec:system-model} 
reproduces the system model from~\cite{adya1999weak}, which is used to define weak isolation levels 
in~\Cref{sec:weak-isolation}, before the mixing-correct theorem is defined in~\Cref{sec:mixing}.

\subsection{System Model}
\label{sec:system-model}

In Adya's system model, transactions consist of an ordered sequence of read and write operations to an 
arbitrary set of data items, book-ended by a \texttt{BEGIN} operation and a \texttt{COMMIT} or an 
\texttt{ABORT} operation. The set of items a transaction reads from and writes to is termed its 
\emph{item read set} and \emph{item write set}. Each write creates a \emph{version} of an item, which is 
assigned a unique timestamp taken from a totally ordered set (e.g., natural numbers) version $i$ of item $x$ is 
denoted $x_i$; hence, a multiversioned system is assumed. All data items have an initial \emph{unborn} version 
$\bot$ produced by an initial transaction $\tx{T_{\bot}}$. The unborn version is located at the start of each 
item's version order. An execution of transactions on a database is represented by a \emph{history}, $H$. This 
consists of a \emph{partial order of events}, which reflects (i) each transaction's read and write operations, 
(ii) data item versions read and written and (iii) commit or abort operations, and a  \textit{version order}, 
which imposes a total order on committed data item versions.

There are three types of dependencies between transactions, which capture the ways in which transactions can 
\emph{directly} conflict. \emph{Read dependencies} capture the scenario where a transaction reads another 
transaction's write. \emph{Antidependencies} capture the scenario where a transaction overwrites the version 
another transaction reads. \emph{Write dependencies} capture the scenario where a transaction overwrites the 
version another transaction writes. Their definitions are as follows:

\begin{description}
\item[Read-Depends]
  Transaction $\tx{T_j}$ \emph{directly read-depends} (\textsf{wr}) on $\tx{T_i}$ if
  $\tx{T_i}$ writes some version $x_k$ and $\tx{T_j}$ reads $x_k$.
\item[Anti-Depends]
  Transaction $\tx{T_j}$ \emph{directly anti-depends} (\textsf{rw}) on $\tx{T_i}$ if
  $\tx{T_i}$ reads some version $x_k$ and $\tx{T_j}$ writes $x$'s next version after
  $x_k$ in the version order.
\item[Write-Depends]
  Transaction $\tx{T_j}$ \emph{directly write-depends} (\textsf{ww}) on $\tx{T_i}$ if
  $\tx{T_i}$ writes some version $x_k$ and $\tx{T_j}$ writes $x$'s next version after
  $x_k$ in the version order.
\end{description}

Using these definitions, a history can be represented by a \emph{direct serialization graph}, $DSG(H)$.
Nodes correspond to committed transactions and edges mark \emph{direct dependencies}, or conflicts, between 
transactions. Again the direction of these dependencies indicate the apparent order of transactions in a 
serial execution. Anomalies are defined by stating properties about the $DSG$.

To illustrate the difference between an Adya history and a schedule, $s$ from~\Cref{sec:sgt} is given
again with versions accessed by each operation (version order: $[x_0\ll x_1,y_0\ll y_1,z_2\ll z_3,a_0\ll a_4]$). \Cref{fig:dsg} shows $DSG(H)$.
$$ H = w_1 [x_1] \ r_2 [x_1] \ r_2 [y_0] \ w_1 [y_1] \ w_2[z_2] \ w_3[z_3] \ r_3 [x_1] \ r_3 [a_0] \ w_4 [a_4] \ c_1 \ c_3 \ c_2 \ c_4 $$
  
\begin{figure}[ht]
  \centering
    \begin{tikzpicture}[node distance=0cm]
      \node (t1) [vertex,xshift=0cm,yshift=0cm] {$T_1$};
      \node (t2) [vertex,xshift=2cm,yshift=0cm] {$T_2$};
      \node (t3) [vertex,xshift=4cm,yshift=0cm] {$T_3$};
      \node (t4) [vertex, xshift=6cm, yshift=0cm] {$T_4$};
      \draw [edge] (t1) -- node [midway,above] {wr} (t2);
      \draw [edge] (t2) -- node [midway,above] {ww} (t3);
      \draw [edge] (t3) -- node [midway,above] {rw} (t4);
      \draw [edge] (t1) to  [out=-45,in=-135] node [midway,below] {wr} (t3);
      \draw [edge] (t2) to  [out=135,in=45]  node [midway,above] {rw}  (t1);
    \end{tikzpicture}
  \caption{Direct serialization graph representation of $H$.}
  \label{fig:dsg}
\end{figure}

The above \emph{item-based} model can be extended to handle
\emph{predicate-based} operations~\cite{adya1999weak}. Database operations are
frequently performed on set of items provided a certain condition called the
\emph{predicate}, $P$ holds. When a transaction executes a read or write based on a
predicate $P$, the database selects a version for each item to which $P$ applies, this is
called the version set of the predicate-based denoted as $\textit{Vset}(P)$. A
transaction $\tx{T_j}$ changes the matches of a predicate-based read $r_i(P_i)$ if
$\tx{T_i}$ overwrites a version in $\textit{Vset}(P_i)$.

\subsection{Weak Isolation Levels}
\label{sec:weak-isolation}

Using the system model in~\Cref{sec:system-model}, definitions of isolation levels are given via a combination of constraints and the 
prevention of
types of cycles in the $DSG$. In total 11 isolation levels are presented
in~\cite{adya1999weak}. To simplify discussions we consider a subset:
\begin{itemize}
\item \level{Read Uncommitted:} proscribes anomaly \anomaly{Dirty Write (G0)},
  the $DSG$ cannot contain cycles consisting entirely of write-depends edges.
\item \level{Read Committed:} proscribes \anomaly{G0} and anomalies,
  (i) \anomaly{Aborted Read (G1a)}, transactions cannot read data item versions
  created by aborted transactions,
  (ii) \anomaly{Intermediate Reads (G1b)}, transactions cannot read intermediate data item versions, and
  (iii) \anomaly{Circular Information Flow (G1c)}, the $DSG$ cannot contain cycles
  consisting of write-depends and read-depends edges.
\item \level{Repeatable Read:} proscribes \anomaly{G0}, \anomaly{G1}, and \anomaly{Write Skew (G2-item)}, the $DSG$ cannot contain cycles
  containing one or more item-anti-depends edges.
% \item \textbf{Consistent View (PL-2+)} proscribes anomaly G-single, the DSG cannot contain a directed cycle with exactly one anti-depends edges, in addition to preventing G1.
% \item \textbf{Snapshot Isolation (PL-SI)} proscribes anomaly G-S1a, the SSG cannot contain  a write- /read-depends edges between transactions without corresponding start edges. Additionally, proscribing anomaly G-S1b, the SSG cannot contain a cycles containing a single anti-depends edge.
\item \level{Serializable:} proscribes anomalies \anomaly{G0}, \anomaly{G1},
  and \anomaly{G2}, the $DSG$ cannot contain any cycles; this extends the coverage to
  predicate-anti-depends edges.
\end{itemize}

\subsection{Mixing of Isolation Levels}
\label{sec:mixing}

To define a correctness criteria for a mixed DBMS, a $DSG$ variant is used to represent a 
mixed history, referred to as a \textit{mixed serialization graph}, $MSG(H)$. A $MSG$ only includes 
\textit{relevant} and \textit{obligatory} conflicts. A relevant conflict is a conflict that is pertinent to a 
given isolation level, e.g., read-depends edges are relevant to \level{Read Committed} transactions but not 
\level{Read Uncommitted} transactions. An obligatory conflict is a conflict that is relevant to one transaction 
but not the other, e.g., an item-anti-depends edge between a \level{Read Committed} transaction and a 
\level{Serializable} transaction is relevant to the \level{Serializable} transaction and not the \level{Read 
Committed} transaction but still must be included in the $MSG$. Adya defines the edge inclusion rules for an 
$MSG$ as follows:
\begin{enumerate}
\item Write-depends edges are relevant to all transactions regardless of isolation level
  thus always included.
\item Read-depends edges are relevant for edges incoming to \level{Read Committed},
  \level{Repeatable Read}, or \level{Serializable} transactions.
\item Item-anti-depends edges are included for outgoing edges from \level{Repeatable Read} and
  \level{Serializable} transactions.
\item Predicate-anti-depends edges are included for outgoing edges from \level{Serializable}
  transactions.
\end{enumerate}
Now in a mixed DBMS, a history is correct if each transaction is provided the isolation guarantees that pertain 
to its level leading to the \emph{mixing-correct theorem}~\cite{adya1999weak}. \Cref{fig:msg} illustrates the differences between 
DSG and MSG representations of a history with the non-relevant and non-obligatory edges removed. 

\begin{theorem}[Mixing-Correct Theorem]
  A history $H$ is mixing-correct if $MSG(H)$ is acyclic and phenomena \anomaly{G1a} and
  \anomaly{G1b} do not occur for \level{Read Committed}, \level{Repeatable Read}, and \level{Serializable} transactions.
\end{theorem}


% An MSG is useful for determining correctness if \level{PL-1} and
% \level{Read Committed} \emph{know} what they are doing and do not modify the database state in
% an inconsistent manner.

\begin{figure}[ht]
  \centering
    \begin{tikzpicture}[node distance=0cm]
      \node (t1) [vertex,xshift=0cm,yshift=0cm] {$T_1$};
      \node [below of=t1,yshift=-0.8cm] {\small{\texttt{\textcolor{black}{Serializable}}}};
      \node (t2) [vertex,xshift=2cm,yshift=0cm] {$T_2$};
      \node [above of=t2,yshift=0.8cm] {\small{\texttt{\textcolor{black}{Read Committed}}}};
      \node (t3) [vertex,xshift=4cm,yshift=0cm] {$T_3$};
      \node [below of=t3,yshift=-0.8cm] {\small{\texttt{\textcolor{black}{Read Uncommitted}}}};
      \node (t4) [vertex,xshift=6cm,yshift=0cm] {$T_4$};
      \node [above of=t4,yshift=0.8cm] {\small{\texttt{\textcolor{black}{Read Committed}}}};
      \draw [edge] (t1) -- node [midway,above] {wr} (t2);
      \draw [edge] (t2) -- node [midway,above] {ww} (t3);
    \end{tikzpicture}
  \caption{Mixed serialization graph representation of $H$.}
  \label{fig:msg}
\end{figure}
