\section{Conclusion}
\label{sec:msgt-conclusion}

In this paper we presented mixed serialization graph testing, a graph-based scheduler that 
leverages Adya's mixing-correct theorem to permit transactions to execute at different isolation levels.
When workloads contain transactions running at weaker isolation levels, MSGT is able to outperform 
serializable graph-based concurrency control by up to 28\%. Additionally, MSGT scales as the number of 
cores is increased, an important property  given modern hardware. Like SGT, MSGT minimizes the number 
of aborted transactions, accepting all useful schedules under the mixing-correct theorem, which greatly 
improves user experience. 
As part of future work we wish to extend our performance 
evaluation to include industry standard benchmarks such as TPCx-IoT~\cite{tpcx-iot} and TPC-C~\cite{tpcc}.
% Additionally, we presented two further optimizations, reduced relevant DFS and 
% early commit that further reduce aborts and decreases latency respectively. 
In summary, this paper 
strengthens recent work refuting the assumption that graph-based concurrency control is impractical.


