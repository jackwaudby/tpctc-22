\section{Mixed Serialization Graph Testing}
\label{sec:msgt}

This section presents mixed serialization graph testing. 
\Cref{sec:msgt-description} describes the protocol design.
% \Cref{sec:msgt-optimizations} discusses optimizations.
\Cref{sec:msgt-discussion} outlines MSGT's advantages and disadvantages. 
Lastly, \Cref{sec:impl-deta} gives implementation details.

\subsection{Protocol Design}
\label{sec:msgt-description}

MSGT blends together the SGT implementation from~\Cref{sec:sgt} with the mixing-correct theorem 
from~\Cref{sec:mixing-theory}. Rather than using the graph data structure in~\Cref{sec:sgt-many-core-opt}
to represent a conflict graph, in MSGT it is used to represent a mixed serialization graph with one addition: nodes
include transaction's desired isolation level. For each operation, conflicts 
are determined using the transaction's isolation level and the edge inclusion rules for a mixed 
serialization graph enumerated in~\Cref{sec:mixing}: when a transaction $T_i$ detects a conflict with $T_j$ it 
is inserted into the mixed graph if the conflict is relevant to $T_i$ or obligatory for $T_j$. No changes are 
necessary to the per-row meta-data as all direct dependencies can be derived 
from the access history. After edge insertion, a cycle check is performed \emph{before} executing the 
operation. If executing the operation would introduce a cycle the offending transaction is aborted. At commit 
time, as in~\cite{DBLP:conf/icde/Durner019} a transaction can commit if and only if it has no incoming edges. 
This simplifies node deletion and provides order preservation. Note, recoverability in no longer ensured as 
under the mixing-correct theorem it is permissible for a \level{Read Uncommitted} transaction to read from a
transaction that subsequently aborts.  

% \subsection{Optimizations}
% \label{sec:msgt-optimizations}

% The algorithm above is sufficient to satisfy the mixing-correct theorem. However, the graph data structure
% does not make a distinction between the types of conflict in incoming and outgoing edge sets, which can lead 
% to transactions unnecessarily aborting due to detection of a cycle that is not applicable to its isolation level. 

% \begin{figure}
  \centering
  \begin{subfigure}[b]{0.3\linewidth}
    \centering
    \begin{tikzpicture}[node distance=0cm]
      \node (t1) [vertex,xshift=0cm,yshift=0cm] {$T_1$};
      \node[align=center] at (0,-0.8) {\scriptsize{\texttt{Read}}\\\scriptsize{\texttt{Committed}}};
      \node (t2) [vertex,xshift=2cm,yshift=0cm] {$T_2$};
      \node[align=center] at (2,-0.8) {\scriptsize{\texttt{Serializable}}};
      \draw [edge] (t1) to  [out=-45,in=-135] node [midway,below] {wr} (t2);
      \draw [edge] (t2) to  [out=135,in=45]  node [midway,above] {rw}  (t1);
    \end{tikzpicture}
    \caption{MSG}
    \label{fig:msg-all}
  \end{subfigure}
  \begin{subfigure}[b]{0.3\linewidth}
    \centering
    \begin{tikzpicture}[node distance=0cm]
      \node (t1) [vertex,xshift=0cm,yshift=0cm] {$T_1$};
      \node[align=center] at (0,-0.8) {\scriptsize{\texttt{Read}}\\\scriptsize{\texttt{Committed}}};
      \node (t2) [vertex,xshift=2cm,yshift=0cm] {$T_2$};
      \node[align=center] at (2,-0.8) {\scriptsize{\texttt{Serializable}}};
      \draw [green,edge] (t1) to  [out=-45,in=-135] node [midway,below] {wr} (t2);
    \end{tikzpicture}
    \caption{$T_1$'s view}
    \label{fig:msg-t1}
  \end{subfigure}
  \begin{subfigure}[b]{0.3\linewidth}
    \centering
    \begin{tikzpicture}[node distance=0cm]
      \node (t1) [vertex,xshift=0cm,yshift=0cm] {$T_1$};
      \node[align=center] at (0,-0.8) {\scriptsize{\texttt{Read}}\\\scriptsize{\texttt{Committed}}};
      \node (t2) [vertex,xshift=2cm,yshift=0cm] {$T_2$};
      \node[align=center] at (2,-0.8) {\scriptsize{\texttt{Serializable}}};
      \draw [red,edge] (t1) to  [out=-45,in=-135] node [midway,below] {wr} (t2);
      \draw [red,edge] (t2) to  [out=135,in=45]  node [midway,above] {rw}  (t1);
    \end{tikzpicture}
    \caption{$T_2$'s view}
    \label{fig:msg-t2}
  \end{subfigure}
  \caption{Mixed serialization graph and transaction's relevant views.}
  \label{fig:relevant-dfs}
\end{figure}

% In~\Cref{fig:msg-all}, the MSG contains a cycle between two transactions, $T_1$ 
% executed at \level{Read Committed} and $T_2$ executed at \level{Serializable}. As can be seen 
% in~\Cref{fig:msg-t1}, the edges relevant to $T_1$ do not form a cycle, but without consideration of edge 
% types it is possible that $T_1$ traverses the \textsf{rw} edge, detects
% the cycle and unnecessarily aborts. In~\Cref{fig:msg-t2}, it is clear that only $T_2$ is required abort in 
% this scenario. This can be solved by storing distinct edge sets for each conflict type and when cycle checking 
% a \emph{reduced relevant DFS} is executed. Reduced relevant DFS only traverses edges relevant 
% to a transaction's isolation level. Thus in~\Cref{fig:relevant-dfs}, $T_1$ would traverse only \textsf{ww} or 
% \textsf{wr} edges and not detect the global cycle in the MSG, i.e., it would only find cycles involving 
% write-depends and read-depends edge, preventing \anomaly{G0} and \anomaly{G1} anomalies.

% Storing edge sets for each conflict type facilitates another optimization that helps reduce latency. 
% The commit rule states: transactions are prevented from committing until they have no incoming edges. In MSGT,
% there is potential that a transaction is prevented from committing due to an incoming edge not 
% relevant to its isolation level. For example, consider $T_{1}$ in~\Cref{fig:relevant-dfs}, a \level{Read Committed} transaction with a 
% single incoming anti-depends \textsf{rw} edge. The transaction running at 
% weaker isolation is being stalled from committing by an obligatory edge.
% With distinct edge sets an \emph{early commit rule} can be employed. 
% Under this rule transactions can commit \emph{with} incoming edges, provided these are not
% relevant to the transaction's isolation level, i.e., $T_{1}$ could commit immediately.
% Informally, this is safe from a correctness stand point as if a transaction has no incoming
% relevant edges then it can never lie on a cycle relevant to its isolation level, e.g., $T_{1}$
% could never be involved in \anomaly{G0} or \anomaly{G1c} cycles. 
% % In the example above, $T_{RC}$ would not unnecessarily waste CPU
% % cycles checking for a cycle that does not and will not exist.

% Care, however, must be taken regards node deletion from the MSG if the early commit rule
% used. To illustrate this, $T_{1}$ is still a member in a cycle for a transaction
% committing at a higher isolation level ($T_{2}$). Removing the node upon commit would allow the
% possibility of missed cycles and violation of the mixing-correct
% theorem. Thus, regardless of isolation level a node can only be scheduled for removal from
% the MSG once \textit{all} incoming edges have been removed. A possible implementation is
% to offload the management of committed but ``in-play'' nodes to a background thread that
% periodically checks if all edges have been removed, at which point the nodes can be deleted
% and recycled; this would allow nodes to be accessed, if needed, during cycle checking
% initiated by other nodes. 


\subsection{Discussion}
\label{sec:msgt-discussion}

As seen in~\Cref{sec:mixing-wild} many DBMSs offer weak isolation levels and 
thus a considerable portion of applications are built atop such guarantees.
Typically, mixed isolation is an after thought as DBMSs chase performance, in
MSGT mixed isolation is a catered for as a \emph{first-class citizen}. Such an approach
provides a higher degree of concurrency and hence performance, whilst also providing
the optimal property of no unnecessary aborts. 
% Reduced relevant DFS further minimizes aborts and 
% allows for quicker cycle detection, as only the relevant portion of the graph needs be searched. 
% The early commit rule should provide lower latency for transactions that require weaker isolation levels.
% Unfortunately under this rule, the real-time commit order may no longer match the serialization order, 
% i.e., order-preservation is violated. However, in the search for high performance this may be a trade-off the database 
% operator is willing to make.

It is worth noting the utility of weak isolation is limited to applications that can tolerate
potentially non-serializable behavior. Additionally, if a workload exhibits low contention or is designed in 
a manner such that anomalies provably do not occur~\cite{DBLP:journals/tods/FeketeLOOS05}, then the additional overhead of 
managing the MSG has little benefit.

\subsection{Implementation Details}
\label{sec:impl-deta}

MSGT was implemented in our prototype in-memory DBMS which has a pluggable
concurrency control module\footnote{\url{https://github.com/jackwaudby/spaghetti}}.
The DBMS has a pool of worker threads and each transaction is pinned to a specific
worker thread for its duration. Each worker thread has an independent workload
generator. Thus when a transaction is committing we repeatedly execute
the commit routine (check for incoming edges).

To ensure high operation throughput under a concurrent workload data
structures use atomic operations were appropriate. For safe memory reclamation in a concurrent environment
epoch-based garbage collection is used~\cite{DBLP:phd/ethos/Fraser04} and nodes' edge sets are recycled after a 
transaction is committed and deleted.

Adya's system model is defined in terms of a multiversion model, but as the MSGT
scheduler allows transactions to optimistically read dirty records, the possibility of
cascading aborts is introduced. Unwinding writes due to cascading aborts lead to
unnecessary system load and which is not useful for both user and system.
Therefore, only one uncommitted transaction is allowed to modify a data item.
Also, the prototype DBMS does not currently support predicate-based operations, thus
an item-based read/write model is assumed. Hence, some isolation levels, e.g.,
\level{Repeatable Read} can not be captured. 