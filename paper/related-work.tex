\section{Related Work}
\label{sec:rel-work}

In recent years, concurrency control research has focused on optimizing protocols for machines with many cores.
In this context, scalability is often limited by global timestamp counters used to generate transaction ids.
Silo~\cite{DBLP:conf/sosp/TuZKLM13} and TicToc~\cite{DBLP:conf/sigmod/YuPSD16} are examples of protocols designed to mitigate against this issue. 
% Optimistic protocols perform well when the conflicts are low, but otherwise introduce a significant 
% number of aborts. 
% In an attempt to combat this issue, mostly-optimistic concurrency control (MOCC) combines 
% 2PL, for high contented rows, with OCC for less contended rows. 
PSSI~\cite{DBLP:conf/icde/RevilakOO11} uses a graph-based scheduler on top of anti-dependencies to 
minimize the unnecessary aborts, but its many-core performance is restricted as it receives a global lock for its certifier graph.
This paper was inspired by the graph-based
approach taken in~\cite{DBLP:conf/icde/Durner019}, which avoids the global timestamp and single-threaded exclusive validation 
phase bottlenecks.
Adya's isolation theory has been used in ELLE~\cite{DBLP:journals/pvldb/AlvaroK20} to experimentally test the isolation guarantees provided by databases. ELLE infers an Adya dependency graph from client-observed  transactions and check for cycles. It is also used in IsoDiff~\cite{DBLP:journals/pvldb/GanRRB020}, a tool which aids 
developers in debugging the anomalies caused by weak isolation.

