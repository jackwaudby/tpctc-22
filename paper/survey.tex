\section{Mixing in the Wild}
\label{sec:mixing-wild}

This section motivates the development of a mixed graph-based scheduler that minimizes unneccessary aborts
by surveying the isolation levels supported by commercial and open source DBMSs.

It is rare for practical DBMSs to offer applications only a singular isolation level, instead permitting
transactions to be run at different isolation levels. In order to assess this claim we surveyed the 
isolation levels offered by 24 DBMSs in~\Cref{tab:survey}\footnote{
\df~Indicates the default setting,
$^{\mathrm{a}}$~Referred to as \level{Read Stability},
$^{\mathrm{b}}$~Behaves like \level{Read Committed} due to MVCC implementation,
$^{\mathrm{c}}$~Implemented as \level{Snapshot Isolation},
$^{\mathrm{d}}$~Requires manual lock management,
$^{\mathrm{e}}$~Behaves like \level{Consistent Read}.}. Classification was performed 
based on each database's public documentation. We found 7 isolation levels represented:
\level{Read Uncommitted}, \level{Read Committed}, \level{Cursor Stability}, \level{Snapshot Isolation}, 
\level{Consistent Read}, \level{Repeatable Read}, and \level{Serializable}. Note, the exact behavior of each 
isolation level is highly system-dependent. Interestingly, we found 18 databases supported multiple isolation
levels. Of systems offering a singular isolation level \level{Serializable} was the most common; these systems 
were typically NewSQL~\cite{DBLP:journals/sigmod/PavloA16} systems, e.g., 
CockroachDB~\cite{DBLP:conf/sigmod/TaftSMVLGNWBPBR20}. This may suggest a trend away 
from mixed DBMSs, however, TiDB recently added support for \level{Consistent Read} isolation~\cite{tidb} 
indicating the utility of weaker isolation in practical systems remains.
% The prevalence of mixed DBMSs supports 
% the development of a highly performant mixed concurrency control protocol that minimizes unnecessary aborts.

\begin{table}[htbp]
  \caption{Isolation Levels Supported by Open Source \& Commercial DBMSs.}
  \begin{center}
  \scriptsize
    \begin{tabular}{|l|c|c|c|c|c|c|c|}
      \hline
      \textbf{Database}&\multicolumn{7}{|c|}{\textbf{Isolation Level}} \\
      \cline{2-8}
      \textbf{System}      &\bi{RU} &\bi{RC} &\bi{CS} &\bi{SI} &\bi{CR} &\bi{RR} &\bi{S}\\
      \hline
      Actian Ingres 11.0   &\y      &\y      &\y      &\nn     &\nn    &\y       &\y\df\\
      Clustrix 5.2         &\nn     &\y\ft{e}&\nn     &\nn     &\nn    &\y\ft{*c}&\y   \\
      CockroachDB 20.1.5   &\nn     &\nn     &\nn     &\nn     &\nn    &\nn      &\y\df\\
      Google Spanner       &\nn     &\nn     &\nn     &\nn     &\nn    &\nn      &\y\df\\
      Greenplum 6.8        &\y\ft{b}&\y\df   &\nn     &\nn     &\nn    &\y       &\nn  \\
      Dgraph 20.07         &\nn     &\nn     &\nn     &\y\df   &\nn    &\nn      &\nn  \\
      FaunaDB 2.12         &\nn     &\nn     &\nn     &\y      &\nn    &\nn      &\y\df\\
      Hyper                &\nn     &\nn     &\nn     &\nn     &\nn    &\nn      &\y   \\
      IBM Db2 for z/OS 12.0&\y      &\y\ft{a}&\y\df   &\nn     &\nn    &\y       &\nn  \\
      MySQL 8.0            &\y      &\y      &\nn     &\nn     &\nn    &\y\df    &\y   \\
      MemGraph 1.0         &\nn     &\nn     &\nn     &\y\df   &\nn    &\nn      &\nn  \\
      MemSQL 7.1           &\nn     &\y\ft{*e}&\nn    &\nn     &\nn    &\nn      &\nn  \\
      MS SQL Server 2019   &\y      &\y\df   &\nn     &\y      &\nn    &\y       &\y   \\
      Neo4j 4.1            &\nn     &\y\df   &\nn     &\nn     &\nn    &\nn      &\y   \\
      NuoDB 4.1            &\nn     &\y      &\nn     &\nn     &\y\df  &\nn      &\nn  \\
      Oracle 11g 11.2      &\nn     &\y\df   &\nn     &\y      &\nn    &\nn      &\nn  \\
      Oracle BerkeleyDB    &\y      &\y      &\y      &\y      &\nn    &\nn      &\y   \\
      Oracle BerkeleyDB JE &\y      &\y      &\nn     &\nn     &\nn    &\y\df    &\y   \\
      Postgres 12.4        &\y\ft{b}&\y\df   &\nn     &\nn     &\nn    &\y\ft{c} &\y   \\
      SAP HANA             &\nn     &\y\df   &\nn     &\y      &\nn    &\nn      &\nn  \\
      SQLite 3.33          &\y      &\nn     &\nn     &\nn     &\nn    &\nn      &\y\df\\
      TiDB 4.0             &\nn     &\nn     &\nn     &\y\df   &\y     &\nn      &\nn  \\
      VoltDB 10.0          &\nn     &\nn     &\nn     &\nn     &\nn    &\nn      &\y\df\\
      YugaByteDB 2.2.2     &\nn     &\nn     &\nn     &\y\df   &\nn    &\nn      &\y   \\
      \hline
    %   \multicolumn{8}{l}{\df~Indicates the default setting.} \\
    %   \multicolumn{7}{l}{$^{\mathrm{a}}$~Referred to as \bi{Read Stability}.} \\
    %   \multicolumn{7}{l}{$^{\mathrm{b}}$~Behaves like \bi{Read Committed} due to MVCC implementation.} \\
    %   \multicolumn{7}{l}{$^{\mathrm{c}}$~Implemented as \bi{Snapshot Isolation}.} \\
    %   \multicolumn{7}{l}{$^{\mathrm{d}}$~Requires manual lock management.} \\
    %   \multicolumn{7}{l}{$^{\mathrm{e}}$~Behaves like \bi{Consistent Read}.} \\
    \end{tabular}
    \label{tab:survey}
  \end{center}
\end{table}